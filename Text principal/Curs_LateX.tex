\documentclass[15pt,twosides,onecolumn,openany]{book}
\usepackage{graphicx} 
\usepackage[catalan]{babel}
\usepackage{emptypage}
\usepackage{hyperref}
\usepackage{mathtools}
\usepackage{blindtext}
\usepackage[utf8]{inputenc}
\usepackage{caption}
\usepackage{subcaption}
\usepackage{wrapfig}
\usepackage[a4paper]{geometry}
\geometry{top=2.5cm, bottom=2.5cm, left=2.5cm, right=2.5cm}
\usepackage{fancyhdr}
\pagestyle{fancy}
\usepackage{amsmath}
\usepackage{amssymb}
\usepackage{amsfonts}
\providecommand{\norm}[1]{\lVert#1\rVert}
\hypersetup{colorlinks=true,urlcolor=blue,linkcolor=blue}
\usepackage{multirow}
\usepackage{multicol}
\usepackage{rotating}
\usepackage{titlesec}

\newenvironment{Figura}
  {\par\medskip\noindent\minipage{\linewidth}}
  {\endminipage\par\medskip}

\titleformat{\section}  % comando de sección a formatear
  {\fontsize{14}{16}\bfseries} % formato para toda la línea
  {\thesection} % cómo mostrar el número
  {0.4em} % espacio entre el número y el texto
  {} 
  [] 

\fancyhf{}
\fancyhead[RO,LE]{\rightmark}
\fancyhead[LO,RE]{\leftmark}
\fancyfoot[RO,LE]{\thepage}
\fancyfoot[LO,RE]{Curs d'Introducció a LaTeX}

\renewcommand{\chaptermark}[1]{\markboth{\thechapter.\ #1}{}}
\renewcommand{\sectionmark}[1]{\markright{\thesection.\ #1}}

\graphicspath{ {images/} }

\begin{document}
\newpage
\thispagestyle{empty}
\begin{titlepage}
    \centering
    \vspace*{\fill}  
    {\Huge \textsc{Curs d'Introducció a \LaTeX{}} \par}
    \vspace{1cm}
    {\Large Optica't UAB \par}
    \vspace{0.5cm}
    {\large Versió 1.0 \par}
    \vspace{5cm}
    \vspace*{\fill} 
\end{titlepage}


\newpage

\thispagestyle{empty}
Benvolguts i benvolgudes a aquest text introductori de LaTeX. Som Optica't, un grup de divulgació d'òptica física format per estudiants de la Universitat Autònoma de Barcelona. Aquest text està pensat per a facilitar l'aprenentatge d'aquesta eina a alumnes que comencin els seus estudis universitaris. Clarament, està orientat a graus científics, el nostre aprenentatge de LaTeX es basa en les necessitats i la curiositat que hem desenvolupat al llarg dels cursos en carreres com física o nanociències. Tot i això, pensem que LaTeX és una eina molt útil per a més disciplines i esperem que molta gent pugui fer profit d'aquest curs.

\newpage
\tableofcontents
\pagestyle{empty}
\newpage
Agraïments
\newpage
\pagestyle{fancy}
\chapter{Introducció}
\thispagestyle{empty}

Dediquem aquest capítol a explicar des de zero com funciona LaTeX i com podem començar a crear els nostres primers documents. La finalitat del capítol és que us quedeu amb els conceptes bàsics de com funciona l'eina, la seva sintaxi, els possibles errors que trobareu, etc.

\section{Què és LaTeX i la seva utilitat}
LaTeX és un editor de textos que funciona de forma similar a un llenguatge de programació compilat. Està basat en el llenguatge de baix nivell \TeX{}, i està pensat principalment per a escriure textos científics de forma senzilla i elegant. La filosofia d'aquesta eina és que qui redacti no es preocupi del format del document, només en què ha d'escriure. A diferència d'un editor de textos \textsc{WYSIWYG} (en anglès, 'el que veus és el que obtens') al treballar amb LaTeX no podem veure el resultat del que redactem al moment d'escriure-ho. En LaTeX es treballa directament amb codi i després un compilador ho interpreta i genera un resultat (per exemple imprimeix el document per pantalla o genera un arxiu \texttt{.pdf}).\\\\
En un principi pot semblar que escriure en codi i no veure què és el que estem generant és contraproduent i pot fer menys atractiva la idea de treballar amb aquest editor i no amb altres més populars. Però, com veureu al llarg d'aquest text, aquest desavantatge queda opacat per la quantitat d'eines útils i de personalització que s'obtenen en treballar amb LaTeX.\\\\
Aquest curs se centra a abastar tot el necessari per a començar a escriure textos científics, com ara lliuraments de diverses assignatures de la carrera o per al treball de fi de grau. Però, en ser només un text introductori, no podrem explicar totes les eines que ofereix LaTeX bàsic i encara menys totes les eines que ha creat la comunitat. La utilitat d'aquest editor de textos va més enllà de textos científics semblants a \textit{papers}, LaTeX té incorporades eines per a escriure llibres, per a escriure música, per a fer presentacions... Us convidem a informar-vos més enllà d'aquest text perquè podeu jugar amb totes les possibilitats que ofereix LaTeX.
\section{Entorns per a treballar amb LaTeX}
Per començar a treballar amb LaTeX primer necessitem saber on! És a dir, necessitem saber qui és el nostre entorn de treball. Igual que passa amb els llenguatges de programació, podem córrer LaTeX a dins d'una gran varietat d'entorns. Cada entorn és lleugerament diferent de l'altre, les seves característiques depenen de la versió i de quin sigui el flux de treball pensat específicament per a aquell entorn. Escollir un bon entorn pot arribar a ser una mica enrevessat, per tant, per a facilitar la tasca a les persones que encara no han fet servir LaTeX o que no estiguin familiaritzades en treballar amb programes de manera local centrarem el curs en l'entorn d'Overleaf. Sense ser exhaustius també parlarem sobre altres entorns com algunes eines d'escriptori o a escriure directament en un document de text amb eines més específiques de programació.\\\\
Per al lector o lectora que ja tingui experiència amb Overleaf i vulgui fer servir les altres eines, pròximament crearem una guia que complementi aquest curs amb més informació, de moment ens centrarem en que pugueu escriure el vostre primer document en LaTeX.
\subsection{Overleaf}
L'entorn per exce\lgem ència per a aprendre LaTeX i per a fer petits projectes co\lgem aboratius és Overleaf. Per a fer-ho servir no cal insta\lgem ar-se res, només necessites connexió a internet i un usuari. Overleaf és un entorn de LaTeX en línia pensat per a fer més fàcil la co\lgem aboració entre usuaris a l'hora de crear un document compartit. Els seus avantatges són:
\begin{itemize}
    \item No requereix cap insta\lgem ació

    \item Interpreta el codi fins i tot si aquest conté errors

    \item Permet treballar conjuntament amb diverses persones\footnote{Actualment la limitació per a editar el mateix projecte d'Overleaf es limita a dos usuaris. La resta d'usuaris que entrin al projecte només poden llegir-ho.}

    \item Permet classificar els teus projectes amb etiquetes

    \item Té insta\lgem ats alguns paquets que el LaTeX pur no té per defecte

    \item Té un sistema de navegació i una visualització del log de compilació intuïtius 

    \item Té integrades algunes eines que faciliten la generació de codi, com ara botons o \textit{shot cuts} que canvien el format del text com en Word o un generador de taules buides

    \item Proporciona un editor visual on es pot treballar en un entorn semblant a un editor WYSIWYG
\end{itemize}
\subsection{LaTeX d'escriptori}
\subsection{Eines alternatives}

\section{Estructura bàsica d'un document}
\subsection{Sintaxi}
\subsection{Tipus de document}
\subsection{Paquets}
\subsection{Entorns}

\chapter{Format de text i estructura}
\thispagestyle{empty}
\section{Format del text pla}
\section{Justificació i espaiat del text}
\section{Llistes}
\section{Múltiples columnes}
\section{Estructura per a un document científic}
\subsection{Pàgina de títol i índex}
\subsection{Capítols, seccions i subseccions}
\subsection{Footnotes i etiquetes}
\subsection{Colorboxes}

\chapter{L'entorn matemàtic}
\section{Formes de cridar un entorn matemàtic}
\section{Sintaxi de l'entorn matemàtic}
\subsection{Operacions bàsiques}
\subsection{Parèntesis i claudàtors}
\subsection{Alguns accents, simbols i formats útils}
\subsection{Límits, sumatoris, productoris i integrals}
\section{Matrius i alineació}
\section{Exemples}

\chapter{Figures i taules}
\section{Figura}
\section{Subfigura}
\section{Taula}
\chapter{La bibliografia}
\section{L'arxiu \texttt{.bib}}
\section{Referenciar i imprimir la bibliografia}

\chapter{Personalització avançada}
\section{Paquets útils}
\subsection{babel}
\subsection{fancyhdr}
\subsection{hyperref}
\subsection{wrapfig}
\subsection{mhchem}
\end{document}